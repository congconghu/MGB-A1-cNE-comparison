\documentclass[11pt]{article}

\usepackage{natbib}
\setcitestyle{authoryear,open={(},close={)}}
\usepackage{fullpage}
\usepackage{setspace}
\usepackage{parskip}
\usepackage{titlesec}
\usepackage[section]{placeins}
\usepackage{xcolor}
\usepackage{breakcites}
\usepackage{lineno}
\usepackage{hyphenat}
\usepackage{comment}
\usepackage{xcolor}


\setlength\columnsep{25pt}

\usepackage{times}

\renewenvironment{abstract}
  {{\bfseries\noindent{\abstractname}\par\nobreak}\footnotesize}
  {\bigskip}

\titlespacing{\section}{0pt}{*3}{*1}
\titlespacing{\subsection}{0pt}{*2}{*0.5}
\titlespacing{\subsubsection}{0pt}{*1.5}{0pt}

\usepackage{graphicx}
\usepackage[space]{grffile}
\usepackage{latexsym}
\usepackage{textcomp}
\usepackage{longtable}
\usepackage{tabulary}
\usepackage{booktabs,array,multirow}
\usepackage{amsfonts,amsmath,amssymb}

\usepackage[utf8]{inputenc}
\usepackage[ngerman,greek,english]{babel}


\begin{document}


\begingroup
\let\center\flushleft
\let\endcenter\endflushleft
\endgroup

%\linenumbers
\doublespacing

\sloppy

\textbf{Title:} \textcolor{red}{TBD (50 word max)}

\textbf{Abbreviated title}: \textcolor{red}{TBD (50 character max)}

\textbf{Author Names and Affiliations:} Congcong Hu\(^1\)\(^2\)\(^3\), Christoph Schreiner\(^1\)\(^3\)

\(^1\)Coleman Memorial Laboratory, \(^2\)Neuroscience Graduate Program, and \(^3\)Department of Otolaryngology-Head and Neck Surgery, University of California-San Francisco, San Francisco, California 94158.

\textbf{Correspondence:} Christoph Schreiner, christoph.schreiner@ucsf.edu

\textbf{Number of pages:} \textcolor{red}{tbd}

\textbf{Number of figures:} 8

\textbf{Number of Words}

\textbf{Abstract:} \textcolor{red}{tbd}

\textbf{Introduction:} \textcolor{red}{tbd}

\textbf{Discussion:} \textcolor{red}{tbd}

\textbf{Conflict of interest statement:}~The authors declare no competing financial interests.

\textbf{Acknowledgements:} \textcolor{red}{tbd}
\begin{comment}
identify all funding sources and may also be used to note intellectual, technical, or other assistance that does not warrant authorship.
\end{comment}

\textbf{Author Contributions:} \textcolor{red}{tbd}



\newpage

\section*{Abstract}
\begin{comment}
(250 Words Maximum)
It should provide a concise summary of the objectives, methodology (including the species studied and whether one or both sexes were included), key results, and major conclusions of the study. It should be written in complete sentences, without subheadings.
\end{comment}


\section*{Significant Statement}
\begin{comment}
(120 Words Maximum)
The Significance Statement should provide a clear explanation of the importance and relevance of the research in a manner accessible to researchers without specialized knowledge in the field and informed lay readers.
\end{comment}


\section*{Introduction}
\begin{comment}
(650 Words Maximum)
The Introduction should briefly indicate the objectives of the study and
provide enough background information to clarify why the study was
undertaken and what hypotheses were tested.
\end{comment}
The function of coordinated neuronal activities, where groups of neurons fire synchronously, in cognitive processes has long been a subject of interest in systems neuroscience \citep{hebb1949organization, konorski1948conditioned}. Initially, they were difficult to observe experimentally, but more recent technological advancements in large-scale recording, such as two-photon imaging and high-density multi-channel probes, have facilitated extensive investigations into the properties and functions of coordinated neuronal firing, primarily within hippocampus and cortical areas \citep{Buzsaki2010, Oberto2021}. These studies yielded evidence of temporal coordination among neurons in distinct brain areas, shedding light on its importance in various cognitive processes, e.g., perception, memory formation, and decision making \citep{baeg2003dynamics, bathellier2012discrete, Bizley2010, boucly2022flexible, domanski2023distinct, harris2003organization, laubach2000cortical, Oberto2021}.

Temporal coordination among neurons in the sensory system has been proposed as a mechanism to enhance information processing \citep{dan1998coding, kreiter1996stimulus, See2018, See2021} and facilitate communication within and between brain regions \citep{zandvakili2015coordinated}. By considering the coordinated firing of neurons, it is feasible to achieve a more reliable and specific representation of stimuli \citep{ebrahimi2022emergent, See2018, See2021, yoshida2020natural}, and to identify the encoding of emergent properties \citep{decharms1996primary, shahidi2019high}. Additionally, studies have shown that elevated coordination in neuronal activities in an output or sender area precede activities in a target or receiver area \citep{zandvakili2015coordinated}. Given that neuronal ensembles have been proposed as the fundamental units for information processing and transmission \citep{Buzsaki2010, yuste2015neuron}, it is crucial to investigate the structure and function of these ensembles at different stages along the sensory pathway.

Of particular interest is the thalamocortical system, which serves as a gateway for sensory information. For example, the auditory cortex contains neurons with temporally coordinated activities, where groups of neurons exhibit correlated firing patterns that contribute to the representation of auditory stimuli \citep{ince2013neural, kreiter1996stimulus, miller2009populations, See2018}. While characteristics of coordinated activities within the cortex have recently been studied \citep{bathellier2012discrete, chamberland2017fast, See2018, See2021}, our understanding of the organization and functional significance of neuronal ensembles in subcortical regions, such as the thalamus, remains limited. The thalamus plays a crucial role in sensory processing as an intermediary between the peripheral sensory system and the cortex \citep{bartlett2013organization}. Notably, the auditory thalamus and cortex demonstrate a high level of interconnectivity, sharing similar and overlapping spectral and temporal response properties \citep{bartlett2007neural, Miller2002}. Considering the strong connection between the thalamus and cortex, an investigation into the shared characteristics of neuronal ensembles in both regions will help us gain a better understanding of the role these ensembles may play in sensory information processing.

In this study, we aimed to identify and characterize coordinated neuronal ensembles (cNEs) in the auditory thalamus and compare them to cNEs in the auditory cortex. Our findings demonstrate the reliable detection of cNEs, defined as groups of neurons exhibiting coordinated activities, in both the medial geniculate body (MGB) and, as confirming previous studies \citep{See2018, See2021} in the primary auditory cortex (A1). The robustness of the applied detection method is evidenced by consistent identification of cNEs across various time bin sizes for both stations. Importantly, we observed a high degree of similarity between cNEs derived from spontaneous and evoked activities, particularly in A1, suggesting that these ensembles represent functional networks that can operate, to a substantial degree, independently of specific sensory stimuli. Furthermore, several shared characteristics were observed between cNEs in MGB and A1. Notably, spikes associated with cNEs conveyed auditory information more reliably than random spikes from the same neurons in both MGB and A1. These findings support the hypothesis that cNEs serve as a ubiquitous mechanism for organizing local networks and function as fundamental units for sensory processing in the brain.

\section*{Materials and Methods}

\subsection*{Animals}
All experimental procedures were approved by the Institutional Animal Care and Use Committee at the University of California, San Francisco (UCSF), and followed the guidelines of the National Institute of Health for the care and use of laboratory animals. Twenty-four female Sprague-Dawley rats (wild type, 250 - 350 g, 2 - 4 months; RRID: MGI: 5651135), sourced from Charles River, were used in this study.

\subsection*{Surgery}
The detailed procedures were described in previous studies \citep{See2018, Homma2020}. Briefly, anesthesia was induced with a combination of ketamine (100 mg/kg, Ketathesia, HenrySchein) and xylazine (3.33 mg/kg, AnaSed, Akorn), along with atropine (0.54 mg/kg, AtroJectSA, HenrySchein), dexamethasone (4 mg/kg, Dexium-SP, Bimeda), and meloxicam (2 mg/kg, Eloxiject, HenrySchein). Additional doses of ketamine (10-50 mg/kg) and xylazine (0-20 mg/kg) were given as needed to maintain anesthesia. Local anesthesia was provided using lidocaine (Lidoject, 2\%, HenrySchein) prior to making incisions. Respiratory rate and depth of anesthesia were continuously monitored and adjusted as needed. The body temperature was monitored and maintained at 37°C using a homeothermic blanket system (Harvard Apparatus 55-7020). Lubricant ophthalmic ointment (Artificial Tears, HenrySchein) was applied to protect the eyes. A tracheotomy was performed to ensure stable breathing during recording. To access the brain, the skin, muscle, skull, and dura over the right temporal lobe were removed, and silicon oil was applied to cover the cortex. A bone rongeur was used to widen the craniotomy window and provide access to the medial geniculate body (MGB) from the top. A cisternal drain was performed to prevent brain swelling.

\subsection*{Electrophysiology}
Electrophysiological recordings were performed using a linear silicon probe with 64 channels (H3, 20$\mu$m channel distance, Cambridge NeuroTech) in the MGB and a 2-shank probe with 64 channels (H2, 25$\mu$m channel distance, Cambridge NeuroTech) in A1. The probes were inserted using microdrives (David Kopf Instruments) at a rate of 25 µm/s to a depth of 4500 to 6000 $\mu$m from the surface of the cortex to reach MGB (Figure 1A) and 900 to 1300 $\mu$m in A1 along the columnar structure (Figure 1B), respectively. Extracellular voltage traces were recorded at a sampling rate of 20 kHz with an Intan RHD2132 Amplifier system (Intan Technologies). Multi-unit (MU) activities (Figure 1) were defined as negative peaks crossing 4 standard deviations (SD) from the mean in the extracellular voltage trace filtered between 300 and 6000 Hz. Single unit (SU) activities were obtained by spike sorting using Kilosort 2.5 \citep{Steinmetz2021, Pachitariu2023}, followed by manual curation using Phy (https://github.com/cortex-lab/phy). Single units on the same electrode were combined to form multi-units (Figure 8). The location of the MGB and A1 were determined based on the tonotopic organization and properties of multi-unit responses to pure tones \citep{Polley2007, Anderson2011, Morel1987}.

\subsection*{Stimuli}
To measure frequency tuning, we presented pure tones with frequencies ranging from 0.5 to 32 kHz in 0.13 octave steps and sound levels from 0 to 70 dB in 5 dB steps (50ms, 5ms ramps). Each frequency-sound level combination was presented once in a pseudo-random order, with an inter-stimulus interval of 250ms. To assess the spectrotemporal receptive fields (STRF), we used a 15-minute dynamic moving ripple (DMR) \citep{Escabi2002}. The DMR consisted of 40 sinusoidal carrier frequencies per octave in the range of 0.5 to 40 kHz, each with a random phase. The carriers were modulated by a spectrotemporal envelope with a maximum spectral modulation rate of 4 cycles/octave, a maximum temporal modulation rate of 40 cycles/s, and a maximum modulation depth of 40 dB. The mean intensity of the DMR was set at 70 dB sound pressure level (SPL). All auditory stimuli were generated using MATLAB (MathWorks) and calibrated using a 1/2-inch pressure field microphone (Type 4192, Brüel and Kjær). The stimuli were delivered contralaterally from the recording site using a closed-field electrostatic speaker (EC1, Tucker-Davis Technologies) at a sampling rate of 96 kHz.

\subsection*{Detecting coordinated neuronal ensembles (cNEs)}
To identify groups of neurons that exhibit co-activation, which we refer to as "cNEs," a method combining principal component analysis (PCA) and independent component analysis (ICA) was used \citep{Lopes-dos-Santos2013, See2018}. We selected a bin size of 10ms as standard  synchronization span because it represents the most appropriate time window to capture the synaptic integration window of most cortical neurons \citep{leger2005synaptic, d2015inhibitory}. First, the individual spike trains of simultaneously recorded neurons were binned and normalized using z-score. Next, the z-scored spike matrix underwent PCA to obtain the eigenvalues of the correlation matrix of spike trains. Significance was assigned to eigenvalues exceeding the upper bounds of the Marčenko-Pastur distribution at the 99.5th percentile \citep{Marcenko1967}, which determined the number of cNEs (Figure 2A-ii). Subsequently, ICA (FastICA) was performed on the subspace spanned by the eigenvectors corresponding to the significant eigenvalues. The resulting independent components (ICs) represented ensembles of neurons, with their contributions to each cNE shown as IC weights (Figure 2A-iii). As the signs of IC weights were arbitrary, for each IC, the direction with the largest absolute weight was rendered positive. The length of each IC was normalized to one, making an IC with equal contribution from all neurons have weights of $1/\sqrt{N}$, where N was the number of neurons in the recording. Neurons with weights over $1/\sqrt{N}$ were referred to as "cNE members" \citep{Oberto2021} (Figure 2A-iv).

To calculate the activity of each cNE, the z-scored spike matrix was projected onto a template matrix obtained by the outer product of the IC representing the cNE. The weights of non-member neurons and the diagonal of the template matrix were set to zero so that only co-activation of multiple cNE members contributed to the cNE activity. A null distribution of cNE activities was obtained by projecting a circularly shifted spike matrix, where the temporal relationship of neurons was disrupted, to the template matrix \citep{See2018}. This process was iterated 50 times, and the threshold of cNE activation was defined as the 99.5th percentile of the null distribution (Figure 2A-v). The spikes of cNE members in the time bin where the cNE activated were referred to as "NE spikes" \citep{See2021, Oberto2021}.

\subsection*{Matching cNEs across different bin sizes}
We used the correlation between IC weights to assess the similarity of cNE patterns across different synchronization windows (i.e., time bin width: 2, 5, 20, 40, 80 and 160ms). To visualize the similarity of cNEs identified using 10ms bins to those identified using other bin sizes (Figure 3A), we chose 10ms reference cNEs, calculated the cNEs for the same recording using other bin sizes, and for each bin size, plotted the cNE whose IC weights were best correlated with the 10ms cNE. To measure the variability of cNE identities across bin sizes (Figure 3B, C), we first matched each cNE to the most similar cNE calculated using 10ms bins (i.e., to the 10ms cNE which had the best-correlated IC weights). The proportion of members consistent with 10ms cNEs was then calculated by dividing the number of members in a cNE that were also identified as members in its matching 10ms cNE by the total number of members in the cNE.

\subsection*{Matching cNEs from two recording segments}
To match the IC weights of cNEs identified from two different segments of a recording session, e.g., adjacent recording blocks (Figure 4), we used an iterative process that involved selecting pairs with the highest correlations (Spearman's r). First, we computed the correlations between all possible pairs of cNEs that were generated from the two different segments. Then, the pair with the highest correlation was set aside, and the same process was repeated with the remaining cNEs until all cNEs were paired. If there were any remaining cNEs that did not have a match due to a difference in the number of cNEs between the two segments, they were left unmatched.

\subsection*{STRF analysis}
For analysis, we down-sampled DMR to a resolution of 0.1 octaves in frequency and 5ms in time. We used the reverse correlation method to obtain the STRFs of the units \citep{Theunissen2000, Escabi2002}. To derive the STRFs, we averaged the spectrotemporal envelopes of the stimulus over a period of 100ms preceding spikes (Figure 1C and 7A). Positive (red) values on a STRF indicate that the sound energy at that frequency and time tends to increase the firing rate of the unit, while negative (blue) values indicate where the stimulus tends to decrease the firing rate of the unit. The frequency of the highest absolute value on the STRF is considered to be the best frequency (BF) of the unit \citep{Miller2002}.

We used mutual information (MI) as the metric to quantify the amount of information we can obtain about the stimulus by observing spikes of a unit \citep{Atencio2008, See2018}. The stimulus segment \emph{s} preceding each spike was projected onto the STRF via the inner product \emph{z} = \emph{s} * STRF. The projection values were then binned to get the probability distribution \emph{P(z$|$spike)}. The \emph{a priori} distribution of stimulus projection values, \emph{P(z)}, was calculated by projecting all stimulus segments of DMR onto the STRF, regardless of spike occurrence. Both distributions \emph{P(z)} and \emph{P(z$|$spike)} were normalized relative to the mean $\mu$ and standard deviation $\sigma$ of \emph{P(z)}, by \emph{$x = \frac{(z - \mu)}{\sigma}$}, resulting in \emph{P(x)} and \emph{P(x$|$spike)}. The MI between STRF projection values and single spikes was computed according to $I = \int dxP(x|spike)log2\frac{P(x|spike)}{P(x)}$. 

\subsection*{Quantify slow oscillations in neural activity}
To determine whether the neural activity in a recording show prominent pattern of slow oscillations, we used silence density and the coefficient of variation (CV) of MU firing rate as measurements. Silence density was defined as the fraction of 20-ms time bins with no population activity (zero spikes) in the 15 min recording of stimulus-driven or spontaneous neural activities \citep{Mochol2015}. The CV of MU firing rate was calculated as the following: $CV = \sigma / \mu $, where $\mu$ is the mean population firing rate and $\sigma$ is the standard deviation of the population firing rate binned at 20ms time bins. 

\subsection*{Permutation test}
We used this method to determine the statistical significance of differences in cross-correlograms (CCGs) among neurons based on their membership (Figure 2B and 3D), as well as to assess the difference in the proportion of stable cNEs (Figure 4E and 8C). For example, to assess the difference in CCGs between member pairs and non-member pairs in Figure 2B, we shuffled the membership labels of the CCGs and calculated the difference between the average CCGs of member and non-member pairs. We repeated this process 10,000 times to generate null distributions of the CCG difference for each data point. We considered consecutive data points from 0ms-lag with p $<$ 0.05 as significant. To assess the difference in the proportion of stable cNEs, we shuffled either the region label (MGB or A1) or the stimulus condition label (spon, dmr, or cross) and repeated this process 10,000 times to generate a null distribution of the difference in proportion. The significance level was then determined based on the null distribution.

\subsection*{Statistics}
Statistical analysis was performed using Python. To compare two unpaired groups (e.g., Figure 6C), we used Mann–Whitney U tests. To compare two paired groups (e.g., Figure 5A), we used Wilcoxon signed-rank tests. To compare the effect of two independent variables (e.g., Figure 2C), we used two-way ANOVAs and used rank tests with Bonferroni correction for \emph{post hoc} comparisons. To compare two categorical variables, we used Fisher's exact test (e.g., Figure 6D). Permutation tests (e.g., Figure 2B and Figure 4E) and Monte Carlo methods (e.g., Figure 4D) were used as noted. The specific applications of these tests are explained in the results section and figure legends. Significance levels are noted as * (p $<$ 0.05), ** (p $<$ 0.01) and *** (p $<$ 0.001). 


\section*{Results}
\begin{comment}
This section should present the experimental findings clearly and succinctly. Only results essential to establish the main points of the work should be included.

Numerical data should be analyzed using appropriate statistical tests described in the Experimental Design and Statistical Analysis section. In the Results section, authors must provide detailed information for each statistical test applied including degrees of freedom and any estimates of effects size, should be reported in the Results section. Report exact p values rather than ranges (e.g., p = 0.026 rather than p < 0.05). There are many types of analyses that can be reported, but examples include F values (F(1, 72) = 14.5, p = 0.003, ANOVA), t values (t(10) = 2.98, p = 0.043, paired t-test), coefficient of determination (R2), and Bayes factors.
\end{comment}

\subsection*{Auditory responses in MGB and A1}
We conducted extracellular recordings using high-density silicon probes in the rat MGB and A1 (Figure 1A and B). Specifically, we used a 64-channel linear probe to target the MGB, allowing us to cover its span on the dorsal-ventral axis. To target the A1, we used a 2-shank probe with 64 channels, allowing us to record neural activities from adjacent cortical columns. To obtain the tonal response properties of the recording sites in MGB and A1, we presented pure tones of various frequencies and intensities. In the MGB, we observed a frequency gradient in MU responses from low to high along the dorsal-ventral axis, which varied gradually (Figure 1A-i) or abruptly (Figure 1A-ii) depending on the probe's location. Responses on most channels exhibited clear frequency tuning (between the red lines in Figure 1A), which likely originated from the ventral MGB, the primary input station to the A1. We included all single units (SU) from the MGB in our analysis after spike sorting, without distinguishing between sub-regions. In the A1, we found that the MU responses to pure tones from the two shanks of the probe exhibited similar frequency tuning, as the probe landed in adjacent cortical columns (Figure 1B). The responses on each shank showed small variation in their frequency preference along the depth of the probe, as neurons in the same cortical column have consistent characteristic frequencies across the active middle and deep cortical layers \citep{Merzenich1975, atencio2010laminar}. Overall, the recorded neural populations in the MGB had a wide frequency span, while those in A1 exhibited a much narrower range of frequency preferences.

To estimate the STRFs of SUs, we used a 15-minute DMR stimulus, which is a broadband noise with varying spectral and temporal modulation rates \citep{Escabi2002}. The STRFs of MGB neurons showed a clear gradient in frequency preference from low to high along the dorsal-ventral axis (Figure 1C), consistent with the MU responses to pure tones. We then examined the firing correlations between pairs of single units. Within the recorded neural population in the MGB, pairs of neurons showed widely different correlations in their firing activity, even if they were close in proximity and had similar STRFs, similar to what was observed in the cortex \citep{See2018, Mogensen2019, Wahlbom2021}. For example, Neurons \#1 to \#3 had similar receptive fields (Figure 1C). While neuron \#1 and \#3 showed correlated firing in both stimulus-driven and spontaneous activities, neuron \#2 and \#3 showed no positive correlation in their activity despite fairly similar STRFs (Figure 1D). Although the role of neuronal coordination in information processing in the cortex has been extensively proposed and studied \citep{Paninski2004, Buzsaki2010, Bizley2010, Carrillo-Reid2015, See2018}, less is known about the organization of neuronal ensembles in subcortical regions. Therefore, we aimed to identify clusters of neurons that exhibit consistent synchronized firing in the MGB and compare the properties of these ensembles with those in A1.

\subsection*{Groups of neurons in the MGB exhibit coordinated firing, similar to those observed in A1}
To identify coordinated neuronal ensembles (cNEs), which are groups of neurons with synchronous firing, we performed a combined principal- and independent-component analysis (PCA-ICA) \citep{Lopes-dos-Santos2013}. To demonstrate the procedure for detecting cNEs in a population of neurons, consider a recording of spontaneous activity from the MGB as an example (Figure 2A). Among the 20 isolated single units in the recording, some groups of neurons had highly correlated firing with each other, as shown in dark red in the correlation matrix, while others showed little correlation (Figure 2A-i). We performed PCA on the correlation matrix of 10ms-binned spike trains, resulting in 20 eigenvalues and corresponding eigenvectors or PCs (Figure 2A-ii). These eigenvalues describe the contribution of each PC to the variance in the neural population activities. To determine the significance of the patterns extracted by PCA, we compared the eigenvalues to a threshold drawn based on the Marčenko-Pastur distribution (Figure 2A-ii). The Marčenko-Pastur distribution is a theoretical distribution of eigenvalues for the null hypothesis that the spike trains are independent random variables \citep{Peyrache2010, Lopes-dos-Santos2011, Oberto2021}. In this example recording, we observed four significant eigenvalues above the threshold, indicating the presence of four cNEs in the recorded population. Although PCA efficiently extracts ensemble patterns, it has limitations due to its variance maximization framework. When two ensembles account for similar variance in the data on their corresponding axis, the first PC will represent the average of the two instead of an individual ensemble. This problem is even more pronounced when ensembles share neurons. To overcome this limitation, we applied ICA to the subspace spanned by the significant PCs (Figure 2A-iii). This approach is not constrained by the orthogonality requirement of PCA, allowing for a more precise identification of individual cNEs. After the PCA-ICA procedure, we obtained weights of neurons on the axes that define cNEs in the neural population, which were color-coded as columns in Figure 2A-iii. Neurons were considered members of a cNE if their IC weights were higher than what would be expected from even contributions from all neurons (Figure 2A-iv). The activity resulting from co-activation of cNE members can be obtained by projecting the spike matrix to the corresponding IC weights of the cNE. To determine the significance of cNE activity magnitude, we generated a null distribution of the cNE activity values by circularly shuffling spike trains and set the significance criteria at 99.5\% (Figure 2A-v). For example, when cNE \#1 was active, multiple member neurons (2-5 out of 6) fired together. The combined PCA-ICA approach provided a useful framework to investigate the organization and function of coordinated neuronal activity in the MGB and A1.

To provide evidence that cNEs captured groups of neurons with correlated firing, we cross-correlated the spike trains of neurons based on their membership. We compared pairs of neurons that participated in the same cNE (member pairs) to pairs of neurons that did not share membership of the same cNE (non-member pairs) in MGB and A1. In MGB, we observed significantly higher correlation between cNE member pairs compared to non-member pairs within the [-50, 42] ms lag window (shaded red area in Figure 2B-i, bottom left). Similarly, in A1, we observed significantly higher correlation between cNE member pairs compared to non-member pairs within the [-41, 46] ms lag window (shaded blue area in Figure 2B-ii, bottom left). To further confirm that cNE members had synchronized firing on a fine time scale and on a recording-to-recording basis, we binned spike trains in 10ms time bins and compared the average correlation of member and non-member pairs of neurons (Figure 2C). Our analysis revealed that member pairs exhibited significantly higher correlations compared to non-member pairs, while there was no significant difference observed between the two brain regions (Two-way ANOVA: interaction, F(1, 98) = 0.17, p = 0.677; members vs non-members, p = 8.5e-23; MGB vs A1, p = 0.159). These results provided evidence that groups of neurons with coordinated firing exist in MGB, as in A1, and that their coordination was captured by the PCA-ICA procedure, resulting in the identification of cNEs.

\subsection*{Variability in cNE identity under different bin sizes}
Previous studies have investigated neuronal synchrony and coordination across a broad range of timescales, from a few milliseconds \citep{lankarany2019, Shahidi2019, El-Gaby2021} to several hundred milliseconds \citep{miller2014visual, tremblay2015attentional, Filipchuk2022}. In these studies, the selection of a specific timescale was influenced by various factors, including the temporal resolution of the employed recording methods, the targeted functional timescale, and the inter-neuronal distance under investigation. To investigate neuronal coordination in our study, we chose a temporal resolution of 10ms. This time scale is relevant for synaptic integration and auditory responses: adult pyramidal cells typically exhibit a membrane time constant of around 10-30ms \citep{koch1996brief, leger2005synaptic, spruston1992, zhang2004maturation}, although this is influenced by factors such as the level of network activity \citep{leger2005synaptic}. Coincident spikes occurring within 10-15ms of each other have been shown to be more effective in driving the firing of cortical neurons \citep{usrey2000synaptic}, emphasizing the functional significance of this timescale. Additionally, cortical neurons exhibit temporal precision in representing sound envelope modulation within the range of tens of milliseconds \citep{hoglen2018amplitude}. The auditory thalamus exhibits even higher temporal resolution in its sound responses \citep{bartlett2007neural}, tracking periodic click sounds with millisecond precision.

To investigate how different choices of timescale affect the identification of cNEs, we obtained cNEs using various spike train bin sizes--ranging from 2ms to 160ms (Figure 3). Some cNEs showed consistent IC weights across different time bin sizes, with high correlation to IC weights obtained using 10ms bins (Figure 3A-i). We also observed cNEs that were only consistently identified using time bin sizes within a specific range (Figure 3A-ii). To evaluate the similarity of cNEs identified using different bin sizes to those identified using 10ms bins (10ms cNEs), we calculated the Pearson's correlation of their IC weights (Figure 3B). The majority of cNEs identified using 5-40ms bins had a high correlation with 10ms cNEs, based on their IC weights (proportion of cNEs with IC weights correlation $>$ 0.9 (MGB / A1): 5ms - 79.8 / 72.5\%, 20ms - 87.9 / 71.8\%, 40ms - 65.4 / 53.7\%). When the bin size was greatly different from 10ms, a smaller proportion of cNEs showed high IC weight correlation (2ms - 50.0 / 37.3\%, 80ms - 40.8 / 36.9\%, 160ms - 23.4 / 19.8\%). To better understand the impact of time bin size on cNE identity, we also assessed the change in cNE membership with different bin sizes (Figure 3C). We considered cNEs with more than 80\% joint membership to share identity, which is equivalent to having one different member in a 5-neuron cNE. The majority of cNEs identified using 5-40ms bins shared identity with 10ms cNEs (MGB / A1: 5ms - 77.2 / 75.0\%, 20ms - 82.8 / 77.6\%, 40ms - 63.5 / 59.8\%). Using a much smaller or larger time bin size resulted in a smaller proportion of cNEs sharing identity with 10ms cNEs (2ms - 63.0 / 53.3\%, 80ms - 48.0 / 46.4\%, 160ms - 27.7 / 37.0\%). In summary, small variations in time bin size have limited effect on cNE identity in both MGB and A1.

In cases where differences in cNE membership arise due to different bin sizes, we investigated the firing correlations between neurons that have shifted in or out of the ensemble. We compared the membership of neurons in cNEs identified using 10ms and 160ms bin sizes (10ms cNEs and 160ms cNEs, Figure 3D) and categorized neurons as: members in both 10ms and 160ms cNEs, members only in 10ms cNEs, or members only in 160ms cNEs. Neurons in both 10ms and 160ms cNEs had positive correlations in their firing (Figure 3D-i). Neurons only in 160ms cNEs were positively correlated with neurons in both cNEs, although the correlation was significantly weaker in the [-18, 10] ms window in MGB and the [-11, 9] ms window in A1 (Figure 3D-ii) compared to the correlation among neurons in both 10ms and 160ms cNEs (Figure 3D-i). Furthermore, some members of 10ms cNEs were not identified as members in 160ms cNEs. These neurons showed no significant difference in their correlations with the neurons in both cNEs (Figure 3D-iii) compared to the correlation among neurons in both 10ms and 160ms cNEs (Figure 3D-i). Therefore, using wider bin sizes to identify cNEs results in more neurons being included in the ensemble with weaker correlations in a tight time scale, as well as the omission of some neurons with sharp correlations.

\subsection*{Variability in cNEs structures across spontaneous and evoked activities}
Several studies have shown that cortical neuronal ensembles have stable structures across spontaneous and stimulus-driven activities, suggesting a stable local network organization that is utilized in processing stimulus information \citep{jermakowicz2009relationship, luczak2009spontaneous, See2018, Filipchuk2022}. To investigate if such a property also exists in cNEs in the MGB and A1, we recorded continuous segments of neural activity in the absence of sound (spontaneous, here after “spon”) and during the presentation of the DMR stimulus (“dmr”) and divided each segment into two blocks. We then detected cNEs in each block separately and compared their stability within and across stimulus conditions (“cross”), measured by the IC weights correlation between adjacent blocks (Figure 4A and B). We observed that while some cNEs exhibited high stability across stimulus conditions, with IC weights correlation comparable to that of within the same stimulus condition (Figure 4C-i), others showed structures that were less stable across stimulus conditions than within a stimulus condition (Figure 4C-ii). We determined the significance of the IC weight correlations using null distributions generated by circularly shuffling spikes and found that both examples in Figure 4C were significantly stable across stimulus conditions, although one was more stable than the other (Figure 4D). Within spontaneous or stimulus-driven activities, 85-90\% of cNEs exhibited stable structures on adjacent activity blocks in both MGB and A1 (Figure 4E). However, significantly fewer cNEs with stable structures were found across stimulus conditions in MGB (66.7\%) compared to those in A1 (82.1\%) (Figure 4E). Moreover, in MGB, significantly fewer cNEs were stable across stimulus conditions than within a stimulus condition (spon vs. cross, p = 6e-4; dmr vs. cross, p = 8e-4; permutation test), while no such difference was observed in A1 (spon vs. cross, p = 0.191; dmr vs. cross, p = 0.125; permutation test). cNEs with stable structures across stimulus conditions exhibit smaller IC weight correlations compared to their within-stimulus counterparts (MGB: spon vs. cross, p = 1.2e-5; dmr vs. cross, p = 2.7e-5; A1: spon vs. cross, p = 1.4e-5; dmr vs. cross, p = 6.6e-6; Wilcoxon signed-rank test, Bonferroni correction). However, there was no significant difference in the IC weight correlations of stable cNEs between MGB and A1 (spon, p = 0.378; dmr, p = 0.116; cross, p = 1.0; Mann–Whitney U test, Bonferroni correction). The results provided evidence for the stability of cNEs in the MGB and A1 during both spontaneous and stimulus-driven activities, with some cNEs exhibiting higher stability than others across different stimulus conditions. The finding that fewer cNEs exhibit stable structures across different stimulus conditions in MGB compared to A1 may be attributed to differences in the frequency tuning range of recorded populations in the two regions or a potential variation in the local network organization between them.

To test the possibility of false positive detection of cNEs, we generated shuffled data without any temporal correlation between spike trains. We achieved this by circularly shifting spike trains while preserving the firing properties of individual neurons. We then applied the cNE detection algorithm to the shuffled data using the same criteria as for the real data. Our analysis revealed a significantly lower number of cNEs identified in both spontaneous and stimulus-driven activities in the two brain regions (Figure 5A), suggesting that the chance of false positive detection of cNEs is quite low. Furthermore, any cNEs identified in the shuffled data did not exhibit the same stability across stimulus conditions observed in real data (Figure 5B). In summary, our findings indicate that the detection of cNEs in the MGB and A1 is reliable and not susceptible to false positives. Moreover, the properties of cNEs we observe, such as their stability across stimulus conditions, are genuine and not artifacts of random data.

\subsection*{cNE properties in MGB and A1}
We analyzed spontaneous activities in 34 MGB and 17 A1 penetrations, and identified 115 and 83 cNEs, respectively. In MGB, we identified an average of 3.4 $\pm$ 0.9 cNEs per penetration, while in A1 it was 4.9 $\pm$ 1.5, with more cNEs observed in penetrations with more isolated single units (Figure 6A). The mean cNE size was 4.3 $\pm$ 1.5 members in MGB and 5.1 $\pm$ 2.2 members in A1, which was dependent on the number of isolated neurons per penetration (Figure 6B). To control for this dependence, we only included penetrations with 13-29 neurons in Figure 6C and D, where the number of neurons in the recordings overlapped between MGB and A1. The relative cNE size was 26.1 $\pm$ 8.0 \% of the total number of neurons in MGB and 24.7 $\pm$ 8.7 \% in A1, with no significant difference between the two regions (Figure 6C). Of the 407 neurons isolated in MGB, the majority (78.6\%) belonged to a single cNE, 11.5\% did not belong to any cNE, and 9.8\% belonged to multiple cNEs (Figure 6D, red). Of the 234 neurons isolated in A1, the majority (68.4\%) belonged to a single cNE, 12.8\% did not belong to any cNE, and 18.8\% belonged to multiple cNEs (Figure 6D, blue). The distribution of the number of cNEs each neuron participated in was significantly different between the two brain regions, with more neurons in A1 participating in multiple cNEs. 

Next, we investigated whether cNE members were physically and functionally closer to each other than non-member pairs of neurons. The pairwise distance of cNE members was significantly smaller than that of non-member pairs of neurons in both MGB and A1 (Figure 6E-i). Moreover, the span of cNEs, defined as the longest pairwise distance among all members, was shorter than that of randomly selected groups of neurons in the recording (Figure 6E-ii). The tuning of cNE members was also closer to each other, as the difference in the best frequency (BF) between cNE member pairs was smaller than that of non-member pairs (Figure 6F-i). The frequency span of cNEs, defined as the largest difference in BF among all members, was smaller than that of randomly selected groups of neurons in MGB but not A1 (Figure 6F-ii). This is consistent with the fact that the recordings in MGB covered a neural population with a wide range of frequency tuning, while the neurons in A1 had more similar spectral properties. Our results demonstrate that cNEs exist in both MGB and A1 in similar numbers and relative sizes. These cNEs are composed of neurons that are physically and functionally closer to each other than non-member pairs, suggesting a local circuit organization.

\subsection*{cNEs enhance stimulus information encoding}
Synchronization of neuronal spikes in the cortex has been found to enhance information encoding about a stimulus compared to the participating neurons alone \citep{dan1998coding, See2018, atencio2013auditory}. This effect may be explained by the multiplexed nature of a single spike train, whereby spikes representing different aspects of the stimulus are mixed and can be separated based on their synchrony with other neurons \citep{lankarany2019, See2021}. To investigate whether spikes from individual neurons that participate in cNEs also exhibit differential coding compared to the neuron's entire spike train, we compared the STRFs calculated using all the spikes emitted by a neuron or only the subset of spikes that contributed to cNE events, referred to as NE spikes (Figure 7A). The spike trains were subsampled to ensure an equal number of spikes across conditions. Our analyses revealed that the STRFs of NE spikes exhibit stronger excitatory and inhibitory fields compared to the STRFs of all spikes from the neuron, as evidenced by the larger peak-to-trough difference (PTD) of the NE STRFs, which quantifies the difference between the largest and smallest value in the STRF (Figure 7A). The PTD of NE spike STRFs was higher than that of neuron STRFs in both the MGB and A1, with no significant difference in the PTD gain between the two brain regions (Figure 7B). Given that PTD only considers two extreme values in the STRF, we further evaluated the reliability of NE spikes relative to all spikes in encoding the sound features represented by their STRFs by calculating the mutual information (MI) between the stimulus and the spikes. Our results demonstrate that the NE spike STRFs have higher MI than the STRFs of all neuron spikes, and that the increase in MI observed in NE spikes is similar in magnitude in both MGB and A1 (Figure 7C). Collectively, these findings suggest that NE spikes can enhance information processing by increasing the signal-to-noise ratio and promoting more consistent encoding of certain stimulus features compared to including all spikes from the neuron.

\subsection*{cNE formation do not rely on strong slow oscillation}
Slow-wave oscillations, characterized by alternating periods of large and sustained network activity (UP states) and neural quiescence (DOWN states), are frequently observed in the cortex and thalamus during anesthesia \citep{chauvette2011properties, contreras1996mechanisms, hasenstaub2007state, neske2016slow, sanchez2000cellular, steriade1993thalamocortical, steriade1993slow}. To quantify the level of slow oscillations in the recording, we used two measurements: silence density (SD) and the coefficient of variation (CV) of the MU firing rate. SD represents the proportion of recording time when no spike was fired by the neural population, which is characteristic of the DOWN state in slow oscillations. In brains without strong slow oscillations, the population of neurons fires continuously, resulting in low SD. MU firing rate CV measures the level of variation in the firing rate of the MU, which is high for neurons going through UP-DOWN state cycles, but small for neural populations with less synchronized oscillatory activity. Recordings with high SD and high MU firing rate CV showed prominent slow oscillations, with epochs of synchronous firing of neurons and epochs of quiescence with no spikes (Figure 8A-i). In contrast, recordings with low SD and low MU firing rate CV did not exhibit strong slow oscillations, displaying relatively stable and continuous firing (Figure 8A-ii). Recordings with moderate SD and MU firing rate CV exhibited moments of elevated firing, although not as synchronized as in recordings with strong oscillations (Figure 8A-iii). By utilizing SD and MU firing rate CV, we were able to differentiate recordings without strong slow oscillations from those with strong slow oscillations.

To investigate the relationship between cNE and slow oscillations, we applied the cNE detection algorithm to activities without strong oscillations in response to DMR during 15-minute recordings. Recordings exhibiting low silence density (SD) and coefficient of variation (CV) of the multi-unit (MU) firing rate were selected for the absence of strong slow oscillation (n(animals): MGB = 8, A1 = 18; n(recordings): MGB = 13, A1 = 10). To determine whether cNEs were solely a byproduct of slow oscillations, we compared the number of cNEs detected in these recordings against the expected false positive rate. We found a significantly higher occurrence of cNEs in activities characterized by low SD and MU firing rate CV in both MGB(n(cNE) = 3.9 $\pm$ 1.0, mean $\pm$ SD; p = 2.4e-4, Wilcoxon signed-rank test) and A1 (n(cNE) = 5.2 $\pm$ 1.9, p = 0.002) when compared to shuffled data (as in Figure 5), supporting that cNE is not a byproduct of slow oscillation.

Slow oscillations in thalamic and cortical firing rates are commonly observed and can be related to synchronized and desynchronized states of the system \citep{metherate1993nucleus, steriade1993thalamocortical, cowan1994spontaneous, sanchez2000cellular, hasenstaub2007state}. The effect of firing rate changes on the information carried by cNEs, however, is not known. Therefore, we tested whether NE spikes exhibit enhanced information encoding in recordings without strong oscillations in response to stimulus. The results showed that NE spike STRFs have higher MI compared to all spikes in both MGB and A1 (Figure 8B), indicating that enhanced information encoding is not specific to synchronized states under slow oscillations. Additionally, a subset of recordings did not show strong slow oscillations in either stimulus-driven or spontaneous activities (n(animals): MGB = 4, A1 = 6; n(recordings): MGB = 5, A1 = 8). We examined the stability of cNEs across and within stimulus conditions in these recordings. The majority of cNEs were stable both within and across stimulus conditions, with no significant difference observed between brain regions or activity blocks due to the limited number of samples (Figure 8C). In summary, cNEs exist in both MGB and A1 without strong slow oscillations in the neural activities. Their enhanced information encoding, and stability across stimulus conditions are not due to a special behavior of neurons in synchronized states under slow oscillations.

\section*{Discussion}
\begin{comment}
(1500 Words Maximum)
The discussion section should be as concise as possible and include a brief statement of the principal findings, a discussion of the validity of the observations, a discussion of the findings in light of other published work dealing with the same or closely related subjects, and a statement of the possible significance of the work. Extensive discussion of the literature is discouraged.
\end{comment}
The aim of this study was to test the hypothesis that cNEs with enhanced information processing exist in the auditory thalamus and are broadly similar to those observed in the auditory cortex. Our results demonstrate that cNEs exist in MGB and A1 during both spontaneous and stimulus-driven activities. These cNEs exhibit consistent patterns across different bin sizes and share similar properties between the two brain regions. Importantly, we observed that spikes in coordination with other member neurons, referred to as NE spikes, displayed higher reliability and conveyed more stimulus-related information compared to random spikes originating from the same neuron. Furthermore, cNEs cannot be attributed to false positive detection by the cNE algorithm and their characteristics are not mere byproducts of slow state oscillations commonly observed in anesthetized animals. These findings support the concept of neuronal ensembles as a general rule of local organization for information processing, at least in the central sensory system.

\subsection*{cNEs are ubiquitous in local organization}
Neuronal ensembles have long been proposed as fundamental units for information processing and transmission in the brain \citep{Buzsaki2010, hebb1949organization}. This idea is supported by evidence of precise temporal coordination among neurons within cortical columns \citep{atencio2013auditory, lankarany2019, See2018}. Cortical columns are composed of neurons with relatively homogeneous properties that are maintained through intracortical processing and shared afferent input, exhibiting distinct boundaries between columns \citep{mountcastle1997columnar}. This raised the question of whether these coordinated neuronal ensembles in auditory cortex, characterized by precise temporal coordination \citep{See2018}, are unique to cortical organization due to the high anatomical and functional interconnectivity and shared input within a column, or if they represent a general organizational and information processing unit across auditory pathways. One crucial station along the auditory pathway, closely associated with cortical function, is the thalamus, which serves as the primary source of ascending input to the primary auditory cortex \citep{smith2012thalamocortical, winer2005auditory}. Neurons in the MGB exhibit similar spectral and temporal properties to those in A1 \citep{Miller2002, bartlett2007neural}. Furthermore, there is reciprocal connectivity between MGB and the auditory cortex, with convergence of frequency tuning and spectral and temporal modulation preferences, preserving topographic organization in both regions \citep{read2011distinct, miller2001functional}. Therefore, investigating neuronal coordination in the auditory thalamus, where neurons possess similar properties but differ in their organizational and cytoarchitectonic patterns from auditory cortex \citep{winer2010neurochemical}, can provide insights into whether cNEs are general organizational principles of local circuits or specialized units specific to the local circuit composition.

To investigate the presence of cNEs in the auditory thalamus, we conducted neural recordings across multiple iso-frequency layers of the MGB and from adjacent columns in the A1. Our findings provide evidence for the existence of cNEs in both the MGB and the A1 at a similar temporal resolution of 10ms (Figure 2). Despite the differences in the frequency span covered by the recorded neurons in MGB and A1, we consistently identified cNEs with similar properties. In both regions, each cNE comprised approximately 25\% of the simultaneously recorded neurons within a given region (Figure 6C). In addition, neurons belonging to the same cNE exhibited closer spatial proximity and shared more similar tuning properties (Figure 6E and F), indicating functional coherence within these ensembles. It is important to acknowledge that our recordings were limited to relatively small populations of 10-40 neurons due to the techniques employed. Therefore, the relative cNE size, as well as the confinement of spatial and frequency tuning properties within cNEs may vary when larger populations with hundreds or thousands of neurons were recorded.

We also demonstrated that the identification of cNEs relies on the temporal coordination among neurons. To assess the impact of temporal relationships, we performed circular shuffling of spike trains, thereby disrupting the original temporal order among neurons. Significantly fewer cNEs were identified in both MGB and A1 when analyzing the shuffled data (Figure 5A). Importantly, these identified cNEs in the shuffled data lacked the key properties observed in the cNEs identified in the real data, such as stability across different stimulus conditions (Figure 5B). This provides support for the crucial role of temporal coordination in the formation and characterization of cNEs in the auditory thalamus and cortex. Furthermore, it is important to note that while cNEs were observed in both brain regions, their presence cannot be generalized to all neural populations, as the temporal relationship among neurons is crucial for their formation and identification.

\subsection*{Temporal frame of cNEs}
The temporal frame employed in identifying neuronal ensembles plays a critical role in shaping our understanding of their nature and function \citep{Buzsaki2010}. In the context of auditory processing, where information rapidly changes within tens of milliseconds \citep{lewicki2002efficient, rosen1992temporal}, we specifically opted for a temporal resolution of 10ms. This choice aligns with the timescale at which the sensory information under investigation operates and holds relevance for synaptic integration. Selecting an appropriate time scale is essential for future investigations into the functional role of cNEs in synaptic transmission within the auditory thalamocortical system.

We have demonstrated the robustness of cNE identification despite slight variations in the time bin sizes used for analysis. This consistency can be attributed to the sparse nature of neural activities, with most of the synchronized neuronal firing occurring at frequencies below 10 Hz \citep{o2010neural}. When inter-spike intervals of neurons predominantly occur at the scale of hundreds of milliseconds, utilizing time windows of 10ms or 20ms has minimal impact on the observed correlations among neurons. However, it is important to highlight that cNEs identified using longer temporal resolutions, such as time bins spanning hundreds of milliseconds, may exhibit significant differences compared to those identified using shorter temporal resolutions in the range of tens of milliseconds. Specifically, cNEs identified with larger time bins may falsely include ‘synchronous’ event from bursting or rebound activity rather than from an initial time period that dominates the transmission of stimulus-triggered information. Long synchronization windows may also include neurons displaying weaker synchronization within short time windows, while potentially excluding neurons with high temporal precision in synchrony. This suggests that the choice of temporal resolution can influence the composition and properties of cNEs, underscoring the importance of considering the appropriate time scale for studying neural ensembles.

Prior research has extensively employed calcium imaging techniques to investigate coordinated neural activities, taking advantage of their ability to capture large cortical areas and simultaneously record from thousands of neurons. However, the temporal resolution of these methods is limited due to the kinetics of calcium sensors. Typically, the integration windows utilized in calcium imaging span hundreds or more milliseconds, which is inadequate for examining precise temporal relationships in neural populations, particularly in the context of auditory responses. Fortunately, new technologies have been developed in recent years, such as ultrafast two-photon fluorescence microscopes \citep{kazemipour2019kilohertz, wu2017ultrafast} and genetically encoded voltage indicators \citep{chamberland2017fast, villette2019ultrafast}, which can be used to further investigate temporal coordination across cortical columns and regions with millisecond resolution. Additionally, the use of high-density probes like neural-pixels probes \citep{jun2017fully}, which offer hundreds of channels, can greatly enhance our exploration of subcortical regions, enabling a more comprehensive understanding of cNEs and their organization within the auditory thalamus.

\subsection*{Stability of cNEs for spontaneous and evoked activities}
Our study revealed that a substantial proportion of cNEs (67\% in MGB and 82\% in A1) maintained consistent composition between spontaneous and sensory evoked neural activity (Figure 4 and 5). This finding suggests that cNEs represent stable configurations within local circuits that can manifest without relying on stimulus-driven synchrony. These results align with previous research demonstrating similarities between patterns observed in stimulus-driven and spontaneous activities \citep{luczak2009spontaneous}. Furthermore, our findings indicate that functional network units are not specific to cortical organization but likely exist as a common modality across multiple stages along the sensory pathway.

The proportion of cNEs with stable structures between spontaneous and evoked activities is significantly lower in the thalamus compared to the cortex. This discrepancy may be attributed to the differences in the frequency range captured by the recordings in these two regions. In the presence of stimulus-driven synchrony, neurons with similar frequency preferences are more likely to fire together. Since cNEs identified in cortical columns during spontaneous activities already consist of neurons with similar receptive fields, the modification of cNE structures is expected to be minimal. In contrast, the frequency range of cNE member neurons in the MGB is broader due to the larger frequency coverage in the recordings (Figure 6F-ii). As stimulus-driven activity increases the correlation among neurons with similar receptive fields, this subset of neurons will be more likely to form cNEs. It is important to exercise caution when extrapolating the relationships observed among neurons with different frequency tuning in the thalamus to the networks of different cortical columns. Notably, a previous study utilizing calcium imaging \citep{Filipchuk2022} observed that under anesthesia, population activities in the cortex displayed high similarity between spontaneous and evoked activity, while greater variability was observed in thalamic activities. Future studies using recordings from multiple cortical columns with millisecond time resolution are needed for a better understanding of the circuit organization and information flow along sensory pathways.

\subsection*{cNEs enhance stimulus encoding}
Considering that cNEs were observed in both spontaneous and stimulus-driven activities, some argue that they are simply a reflection of background activity and not involved in stimulus encoding, if not impairing it \citep{abbott1999effect, jermakowicz2009relationship, zohary1994correlated}. If this were the case, spikes associated with cNE events in a single neuron would be randomly selected from the entire spike train, regardless of the presence of a stimulus. In such a scenario, these spikes would carry a similar or even lower amount of information compared to all the spikes emitted by the neuron. Contrary to this assumption, our observations reveal that NE spikes exhibit a higher signal-to-noise ratio and convey more information per spike when compared to the entire spike train (Figure 7B and C). This finding suggests that cNE events are more stimulus-selective than the contributing neurons \citep{See2021} and exhibit a more reliable response to the stimulus features represented by the cNE STRF. Furthermore, the stable connectivity pattern revealed by spontaneous, intrinsic activity likely represent aspects that have been imprinted by extensive experience and behavioral relevance of the associated functional preferences.

Correlated spikes have the potential to enhance the transmission of salient auditory information when they synchronously converge onto their targets \citep{stevens1998input, zandvakili2015coordinated}. In addition, multiple studies have revealed that individual neurons exhibit a multiplexed nature, wherein spikes from the same neuron can carry information related to different aspects of the stimulus \citep{lankarany2019, See2021, walker2011multiplexed}. It is possible that one of the functions of cNEs is to selectively choose spikes from their constituent neurons that are relevant to each other, effectively propagating the represented information to downstream targets while excluding irrelevant spikes. This mechanism can significantly enhance both the robustness and capacity of information encoded within a population of neurons \citep{See2021, walker2011multiplexed}. Since information encoding relies on the collective activity of a group of neurons, the failure of a single neuron does not disrupt the overall information being encoded and transmitted. Moreover, the combination of neuronal activity patterns allows for a higher information encoding capacity compared to the firing patterns of individual neurons alone. This suggests that the presence of cNEs and their coordination within a neuronal population can facilitate efficient information processing and transmission in the auditory system. Future studies involving simultaneous recordings from two stations along the auditory pathway will be necessary to test this hypothesis.

\subsection*{cNE formation does not depend on strong slow oscillations}
Slow oscillations are commonly observed in the neural activities of anesthetized animals \citep{chauvette2011properties, dasilva2021modulation}. Additionally, it has been demonstrated that the level of slow oscillation in neural activity influences the stimulus-encoding properties of neurons \citep{pachitariu2015state}. Consequently, concerns have been raised regarding whether cNEs are merely a result of the synchrony induced by anesthesia. However, our research findings provide evidence against this notion. In our study, we specifically focused on a much smaller time scale of synchronization that was distinct from the slow oscillations induced by anesthesia. Furthermore, we successfully detected cNEs in recordings without strong slow oscillations. As discussed previously, these cNEs exhibited stable structures and enhanced information properties. This suggests that cNEs are not solely a byproduct of anesthesia-induced synchrony but rather represent a distinct phenomenon with its own characteristics. While we ruled out slow oscillations induced by anesthesia as the major force underlying cNE formation, it is important to note that cNEs may also interact with other types of oscillatory activities, such as gamma rhythms \citep{Oberto2021}. Further research is necessary to investigate the interplay between cNEs and various oscillatory activities in the brain. Such investigations will contribute to a more comprehensive understanding of the neural dynamics underlying the formation of cNEs and their role in information processing and transmission.

\section*{Figure Legends}

\subsection*{Figure 1. \emph{In vivo} recordings in rat MGB and A1}
(A) Left: Schematic of the recording setup in the MGB using a linear 64-channel probe. Ai and Aii: two electrode penetrations with multi-unit (MU) recordings from the MGB. Left: Stacked firing rate (color coded) of pure-tone frequency response areas. Right: Characteristic frequencies (CF, the frequency at which the response threshold is the lowest). The red dashed lines indicate the potential boundaries of the ventral MGB.
(B) Left: schematic of the recording setup in A1 using a 2-shank probe with 64 channels. Right: MU responses to pure tones as in (A).
(C) Example STRFs of SU from a recording in the MGB. Unit number 1-3 indicate the positions and STRFs of pairs of neurons whose CCGs are plotted in (D).
(D) Example CCGs from two pairs of neurons (\#1-\#3 and \#2-\#3). The black bars represent the CCGs of stimulus-driven activities, while the grey lines represent the CCGs of spontaneous activities. The baseline is estimated by averaging the counts in 5ms windows at the shoulders of the CCGs and is indicated by dashed red lines.

\subsection*{Figure 2. Groups of neurons with coordinated activities exist in MGB and A1}
(A) Procedures for detecting cNEs in a thalamic penetration across all layers. (i) Correlation matrix of spike trains. ii) Eigenvalues of the correlation matrix shown in (i). The dashed red line represents the 99.5th percentile of the Marčenko-Pastur distribution, which was used as the significance threshold for eigenvalues. The top four eigenvalues are significant and represent the number of detected cNEs. (iii) IC weights of neurons for each cNE. The green dots represent neurons that are members of a cNE. (iv) cNE members (red stems) are neurons with IC weights exceeding the threshold ($1/\sqrt{N}$) shown as grey areas. (v) Example of cNE activation. Top: Activity trace of cNE \#1. The red dashed line shows the threshold estimated using Monte Carlo methods. The peaks crossing the threshold indicate cNE events when multiple cNE member neurons jointly fire. Bottom: Spike raster of neurons, with red ticks indicating spikes that contribute to instances of cNE events, which were referred to as NE spikes. Shaded areas show member neurons.
(B) Z-scored CCGs (1ms bin) of neuron pairs that were both members of the same cNE (left, members) and neuron pairs that were not members of the same cNE (right, non-members), in MGB (i) and A1 (ii). (i, ii) Top: Stacked CCGs ordered by the peak delay. Bottom: Average of the data above (mean $\pm$ SD; shaded bar: permutation test, shuffling the member/non-member labels).
(C) Average pairwise correlation (10ms bin) from each recording (MGB, p = 1.2e-10; A1, p = 1.5e-5; Wilcoxon signed-rank test, Bonferroni correction).

\subsection*{Figure 3. Variability of IC weights across different time bin sizes.}
(A) Example of two MGB cNEs whose member neurons are either consistently identified across different bin sizes (i) and or only detected using smaller bin sizes (ii). 
(B) Correlation of IC weights obtained using different bin sizes with the IC weights obtained using 10ms bin size. The resulting correlation value between each matching cNE pair is plotted. (Mann–Whitney U test with Bonferroni correction: MGB vs A1, 2ms, p = 0.20; 5ms, p = 0.62; 20ms, p = 0.009; 40ms, p = 0.32; 80ms, p = 1.00; 160ms, p = 1.00).
(C) Proportion of members of a cNE identified for a given bin size that are also members of the 10ms cNE. (Mann–Whitney U test with Bonferroni correction: MGB vs A1, 2ms, p = 0.34; 5ms, p = 1.00; 20ms, p = 0.015; 40ms, p = 1.00; 80ms, p = 1.00; 160ms, p = 1.00).
(D) Top: Schematic of membership of neurons for different bin sizes. Bottom: Mean z-scored CCG (1ms bin, mean $\pm$ SD) of pairs of neurons identified as cNE members on both 10ms and 160ms bins (i), neurons only identified as members on 160ms bin with neurons identified as members on both bin sizes (ii, shaded bar: permutation test, shuffling bin size labels of neuron pairs), and neurons only identified as members on 10ms bin with neurons identified as members on both bin sizes (iii, no time bin showed significance).

\subsection*{Figure 4. cNEs identified in spontaneous activities are largely preserved in stimulus-driven activities.}
(A) Diagram illustrating the recording sequence and partitioning of spontaneous (yellow/orange) and DMR\-evoked (dark green/light green) activity. The four segments allowed the comparison of cNEs obtained within conditions and across conditions.
(B) Absolute correlation values of the IC weights calculated on adjacent recording segments from a MGB recording including the two examples (i and ii) shown in (C).
(C) Example IC weights on adjacent blocks with high (i) and moderate (ii) correlation values across stimulus conditions. The dashed lines show the threshold to determine the membership of the neurons.
(D) The two cNE examples in (C) have significantly matched IC weights across stimulus conditions. See text for how the null distribution was calculated. The significance threshold for the correlation values is set at p = 0.01 (red dashed line). The brown (i) and blue (ii) solid lines represent the two examples in (C). 
(E) Correlation values of IC weights on adjacent activity blocks from all penetrations. The hollow histograms show all correlation values of matched cNEs across stimulus conditions; the histograms show significant correlation values. The inset numbers show the percentage of cNEs with significantly matched IC weights on adjacent blocks in MGB (red) and A1 (blue, permutation test, shuffling the region labels) (MGB: spon vs cross, p = 6e-4, dmr vs cross, p = 8e-4; A1: spon vs cross, p = 0.191, dmr vs cross, p = 0.125, permutation test, shuffling the stimulus condition labels). The triangle arrows show the median of the significant IC weight correlations (two-way ANOVA: interaction, F(2, 410) = 0.93, p = 0.393; MGB vs A1, p = 0.007; stimulus conditions, p = 1.63e-7).

\subsection*{Figure 5. Stable cNEs across stimulus conditions are rarely detected on shuffled data.}
(A) The number of cNEs detected on real data and shuffled data for spontaneous (MGB, p = 1.2e-10; A1, p = 6.1e-5) and stimulus-evoked activities (MGB, p = 4.7e-10; A1, p = 6.1e-5, Wilcoxon signed-rank test with Bonferroni correction).
(B) The correlation of IC weights identified under spontaneous and stimulus-evoked activities on real and shuffled data, in MGB (top) and A1 (bottom). The hollow histograms and histograms show all correlation values and significant correlation values respectively, as in figure 4E. Across stimulus conditions, in MGB, on real data 61\% (52 - 70\%, 95\% confidence interval estimated with bootstrapping) of cNEs have significantly correlated IC weights and are considered stable, while on shuffled data only 2\% (0 - 4\%) con be considered stable. In A1, 88\% (80 - 95\%) of cNEs are stable on real data, while 3\% (2 - 4\%) are stable on shuffled data.

\subsection*{Figure 6. Properties of cNEs in MGB and A1.}
(A) Number of cNEs detected in any given penetration increases with the number of recorded neurons.
(B) cNE size increases with the number of recorded neurons.
(C) The relative cNE size (proportion of member neurons) in MGB and A1 (p = 0.229, Mann–Whitney U test). 
(D) Most neurons belong to only one cNE (MGB 78.6\%, A1 68.4\%). A subset of neurons do not belong to any cNE (MGB 11.5\%, A1 12.8\%) or belong to multiple cNEs (MGB~9.8\%, A1~18.8\%). (Fisher's exact test).
(E) Spatial distribution of cNE members in MGB and A1. (i) Pairwise distance of neurons in the same cNE (colored bar) and neurons not in the same cNE (black line). (ii) Spatial span of cNE members (colored bar) and random groups of neurons with the same number of neurons as cNEs (black line). (F) Frequency tuning distribution of cNE members in MGB and A1. (i) Pairwise distance of neuron's BF in the same cNE and neurons not in the same cNE. BF was the frequency of the highest value on STRF. (ii) Frequency span of cNE members and random groups of neurons with the same number of neurons as cNEs (Mann–Whitney U test).

\subsection*{Figure 7. cNEs can refine sound features encoded by member neurons.}
(A) Two example STRFs of MGB neurons calculated with all spikes (left) and NE spikes (right).
(B) (i) STRF peak-trough difference (PTD) for all spikes or NE spikes only of the neuron (Wilcoxon signed-rank test). (ii) STRF PTD gain in MGB and A1 (p = 0.72, Mann–Whitney U test).
(C) (i) MI between stimulus of all spikes and NE spikes (Wilcoxon signed-rank test). (ii) STRF MI gain in MGB and A1 (p = 0.92, Mann–Whitney U test).

\subsection*{Figure 8. cNE events do not rely on slow oscillation in neural activities.}
(A) Silence density and firing rate coefficient of variation (CV) of population activities in response to DMR. Some recordings show strong slow oscillation in population activities with pronounced silent period between highly active moments (i), while others show little (ii) or moderate (iii) levels of slow MU firing rate oscillation. Recordings with silence density $<$ 0.4 and MU firing rate coefficient of variance $<$ 0.8 (dashed lines) do not show prominent slow oscillation in population activities and are included in (B). 
(B) STRF MI of all spikes from the neuron and NE spikes of the neuron from recordings without prominent slow oscillation (Wilcoxon signed-rank test). 
(C) Correlation values of cNE IC weights on adjacent activity blocks of spontaneous or evoked activity from recordings with no prominent slow oscillations. The inset numbers show the percentage of cNEs with significantly matched IC weights on adjacent blocks in MGB (red) and A1 (blue) (permutation test, shuffling the region labels). The hollow histograms and histograms show all correlation values and significant correlation values with the same presentation scheme as in Figure 4E. (MGB: spon vs cross, p = 0.327, dmr vs cross, p = 0.521; A1: spon vs cross, p = 0.299, dmr vs cross, p = 0.133, permutation test, shuffling the activity blocks labels).

\bibliographystyle{jneurosci}
\bibliography{ref}

\end{document}
